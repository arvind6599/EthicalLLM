\section{Context}

Language models are invariably instilled with an ethical framework, either implicitly via the curated datasets they are trained on, or through explicit methodologies involving specific training and prompting protocols. A notable example of such an ethical guidance framework is the Constitutional AI system developed by the Anthropic team (\cite{constitutionalai}). This methodology is predicated upon the premise of enabling a language model to iteratively refine its responses based on a predefined set of values. These values are subsequently employed to retrain the model using supervised fine-tuning coupled with reinforcement learning, guided by a preference model, as is typical in the Reinforcement Learning from Human Feedback (RLHF) (\cite{rlhf-hf}) paradigm. This method has been demonstrated to significantly enhance the 'harmlessness'—or ethical behavior—and 'helpfulness' of language models. Despite its efficacy, the deontological ethical system that underpins this framework exhibits several notable limitations.

\section{Motivation}
Most ethical guidance systems for language models adhere to deontological principles, as these rule-based systems are straightforward and effective at preventing undesirable outputs like aiding criminal activities. However, as the capabilities and use of language models expand, they encounter increasingly complex ethical dilemmas that challenge simple rules (\cite{edgecase-llm}). This has sparked discussions on the need for more nuanced ethical frameworks that are robust yet abstract enough to avoid alignment with specific moral or political values. Virtue ethics (\cite{sep-ethics-virtue}), focusing on the internal qualities of moral agents, offers a sophisticated alternative that enhances interpretability and ensures truthfulness, making it well-suited to advanced language models.

\section{Goals}
In this project, we sought to incorporate a virtue ethics framework into the model, affecting both the selection of values used to modify responses and the fundamental architecture of the system itself. Virtue ethics emphasizes three critical dimensions of evaluation: the inherent ethicality of the action, the motivation underpinning the action, and the action's effectiveness in fostering virtuous characteristics within the agent (the consequences of doing an action). 
